% Algunas propiedades del documento
\documentclass[12pt,letterpaper,spanish]{article}

% Fuente Times New Roma
\usepackage[T1]{fontenc}
\usepackage{tabto}

%Multi- row
\usepackage{multirow}

% Colores en tablas
\usepackage{xcolor,colortbl}
\usepackage{tabularx}

% Caracteres en español
\usepackage[spanish]{babel}
\selectlanguage{spanish}
\usepackage[utf8]{inputenc}

% Para hacer comentarios
\usepackage{verbatim}

% Hace @ usable
\makeatletter

% Márgenes
\usepackage[margin=2.5cm,bindingoffset=0.5cm]{geometry}
%\newgeometry para cambiar la geometria en cualquier parte del documento

% Imágenes
\usepackage{graphicx}
\graphicspath{ {Z-Imagenes/} }
\usepackage{fancybox}
\usepackage{xcolor}
\usepackage{color}  
\usepackage{blindtext}  



% Para las lineas punteadas en el índice
\usepackage{tocloft}
\renewcommand{\cftsecleader}{\cftdotfill{\cftdotsep}}
\newcommand\mydotfill{\cftdotfill{\cftdotsep}}

% Hyperlinks
\PassOptionsToPackage{hyphens}{url}\usepackage{hyperref}
\hypersetup{
    colorlinks,
    linkcolor={black},
    urlcolor={black}
}

% Para pseudo-codigo
\usepackage{listings}

% Pone numero de pagina a la derecha
\usepackage{scrlayer-scrpage}
\setlength{\headheight}{1.1\baselineskip}
\ifoot[]{}
\cfoot[]{}
\ofoot[\pagemark]{\pagemark}
\pagestyle{scrplain}

% Para que las tablas no se salgan de la sección
\usepackage{float}

\usepackage[final]{pdfpages}
\usepackage{array}

%\setlength\extrarowheight{3pt}   %% put it here for making it global
\setlength\arrayrulewidth{1pt}
\setlength\extrarowheight{3pt}

